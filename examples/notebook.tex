
% Default to the notebook output style

    


% Inherit from the specified cell style.




    
\documentclass[11pt]{article}

    
    
    \usepackage[T1]{fontenc}
    % Nicer default font (+ math font) than Computer Modern for most use cases
    \usepackage{mathpazo}

    % Basic figure setup, for now with no caption control since it's done
    % automatically by Pandoc (which extracts ![](path) syntax from Markdown).
    \usepackage{graphicx}
    % We will generate all images so they have a width \maxwidth. This means
    % that they will get their normal width if they fit onto the page, but
    % are scaled down if they would overflow the margins.
    \makeatletter
    \def\maxwidth{\ifdim\Gin@nat@width>\linewidth\linewidth
    \else\Gin@nat@width\fi}
    \makeatother
    \let\Oldincludegraphics\includegraphics
    % Set max figure width to be 80% of text width, for now hardcoded.
    \renewcommand{\includegraphics}[1]{\Oldincludegraphics[width=.8\maxwidth]{#1}}
    % Ensure that by default, figures have no caption (until we provide a
    % proper Figure object with a Caption API and a way to capture that
    % in the conversion process - todo).
    \usepackage{caption}
    \DeclareCaptionLabelFormat{nolabel}{}
    \captionsetup{labelformat=nolabel}

    \usepackage{adjustbox} % Used to constrain images to a maximum size 
    \usepackage{xcolor} % Allow colors to be defined
    \usepackage{enumerate} % Needed for markdown enumerations to work
    \usepackage{geometry} % Used to adjust the document margins
    \usepackage{amsmath} % Equations
    \usepackage{amssymb} % Equations
    \usepackage{textcomp} % defines textquotesingle
    % Hack from http://tex.stackexchange.com/a/47451/13684:
    \AtBeginDocument{%
        \def\PYZsq{\textquotesingle}% Upright quotes in Pygmentized code
    }
    \usepackage{upquote} % Upright quotes for verbatim code
    \usepackage{eurosym} % defines \euro
    \usepackage[mathletters]{ucs} % Extended unicode (utf-8) support
    \usepackage[utf8x]{inputenc} % Allow utf-8 characters in the tex document
    \usepackage{fancyvrb} % verbatim replacement that allows latex
    \usepackage{grffile} % extends the file name processing of package graphics 
                         % to support a larger range 
    % The hyperref package gives us a pdf with properly built
    % internal navigation ('pdf bookmarks' for the table of contents,
    % internal cross-reference links, web links for URLs, etc.)
    \usepackage{hyperref}
    \usepackage{longtable} % longtable support required by pandoc >1.10
    \usepackage{booktabs}  % table support for pandoc > 1.12.2
    \usepackage[inline]{enumitem} % IRkernel/repr support (it uses the enumerate* environment)
    \usepackage[normalem]{ulem} % ulem is needed to support strikethroughs (\sout)
                                % normalem makes italics be italics, not underlines
    

    
    
    % Colors for the hyperref package
    \definecolor{urlcolor}{rgb}{0,.145,.698}
    \definecolor{linkcolor}{rgb}{.71,0.21,0.01}
    \definecolor{citecolor}{rgb}{.12,.54,.11}

    % ANSI colors
    \definecolor{ansi-black}{HTML}{3E424D}
    \definecolor{ansi-black-intense}{HTML}{282C36}
    \definecolor{ansi-red}{HTML}{E75C58}
    \definecolor{ansi-red-intense}{HTML}{B22B31}
    \definecolor{ansi-green}{HTML}{00A250}
    \definecolor{ansi-green-intense}{HTML}{007427}
    \definecolor{ansi-yellow}{HTML}{DDB62B}
    \definecolor{ansi-yellow-intense}{HTML}{B27D12}
    \definecolor{ansi-blue}{HTML}{208FFB}
    \definecolor{ansi-blue-intense}{HTML}{0065CA}
    \definecolor{ansi-magenta}{HTML}{D160C4}
    \definecolor{ansi-magenta-intense}{HTML}{A03196}
    \definecolor{ansi-cyan}{HTML}{60C6C8}
    \definecolor{ansi-cyan-intense}{HTML}{258F8F}
    \definecolor{ansi-white}{HTML}{C5C1B4}
    \definecolor{ansi-white-intense}{HTML}{A1A6B2}

    % commands and environments needed by pandoc snippets
    % extracted from the output of `pandoc -s`
    \providecommand{\tightlist}{%
      \setlength{\itemsep}{0pt}\setlength{\parskip}{0pt}}
    \DefineVerbatimEnvironment{Highlighting}{Verbatim}{commandchars=\\\{\}}
    % Add ',fontsize=\small' for more characters per line
    \newenvironment{Shaded}{}{}
    \newcommand{\KeywordTok}[1]{\textcolor[rgb]{0.00,0.44,0.13}{\textbf{{#1}}}}
    \newcommand{\DataTypeTok}[1]{\textcolor[rgb]{0.56,0.13,0.00}{{#1}}}
    \newcommand{\DecValTok}[1]{\textcolor[rgb]{0.25,0.63,0.44}{{#1}}}
    \newcommand{\BaseNTok}[1]{\textcolor[rgb]{0.25,0.63,0.44}{{#1}}}
    \newcommand{\FloatTok}[1]{\textcolor[rgb]{0.25,0.63,0.44}{{#1}}}
    \newcommand{\CharTok}[1]{\textcolor[rgb]{0.25,0.44,0.63}{{#1}}}
    \newcommand{\StringTok}[1]{\textcolor[rgb]{0.25,0.44,0.63}{{#1}}}
    \newcommand{\CommentTok}[1]{\textcolor[rgb]{0.38,0.63,0.69}{\textit{{#1}}}}
    \newcommand{\OtherTok}[1]{\textcolor[rgb]{0.00,0.44,0.13}{{#1}}}
    \newcommand{\AlertTok}[1]{\textcolor[rgb]{1.00,0.00,0.00}{\textbf{{#1}}}}
    \newcommand{\FunctionTok}[1]{\textcolor[rgb]{0.02,0.16,0.49}{{#1}}}
    \newcommand{\RegionMarkerTok}[1]{{#1}}
    \newcommand{\ErrorTok}[1]{\textcolor[rgb]{1.00,0.00,0.00}{\textbf{{#1}}}}
    \newcommand{\NormalTok}[1]{{#1}}
    
    % Additional commands for more recent versions of Pandoc
    \newcommand{\ConstantTok}[1]{\textcolor[rgb]{0.53,0.00,0.00}{{#1}}}
    \newcommand{\SpecialCharTok}[1]{\textcolor[rgb]{0.25,0.44,0.63}{{#1}}}
    \newcommand{\VerbatimStringTok}[1]{\textcolor[rgb]{0.25,0.44,0.63}{{#1}}}
    \newcommand{\SpecialStringTok}[1]{\textcolor[rgb]{0.73,0.40,0.53}{{#1}}}
    \newcommand{\ImportTok}[1]{{#1}}
    \newcommand{\DocumentationTok}[1]{\textcolor[rgb]{0.73,0.13,0.13}{\textit{{#1}}}}
    \newcommand{\AnnotationTok}[1]{\textcolor[rgb]{0.38,0.63,0.69}{\textbf{\textit{{#1}}}}}
    \newcommand{\CommentVarTok}[1]{\textcolor[rgb]{0.38,0.63,0.69}{\textbf{\textit{{#1}}}}}
    \newcommand{\VariableTok}[1]{\textcolor[rgb]{0.10,0.09,0.49}{{#1}}}
    \newcommand{\ControlFlowTok}[1]{\textcolor[rgb]{0.00,0.44,0.13}{\textbf{{#1}}}}
    \newcommand{\OperatorTok}[1]{\textcolor[rgb]{0.40,0.40,0.40}{{#1}}}
    \newcommand{\BuiltInTok}[1]{{#1}}
    \newcommand{\ExtensionTok}[1]{{#1}}
    \newcommand{\PreprocessorTok}[1]{\textcolor[rgb]{0.74,0.48,0.00}{{#1}}}
    \newcommand{\AttributeTok}[1]{\textcolor[rgb]{0.49,0.56,0.16}{{#1}}}
    \newcommand{\InformationTok}[1]{\textcolor[rgb]{0.38,0.63,0.69}{\textbf{\textit{{#1}}}}}
    \newcommand{\WarningTok}[1]{\textcolor[rgb]{0.38,0.63,0.69}{\textbf{\textit{{#1}}}}}
    
    
    % Define a nice break command that doesn't care if a line doesn't already
    % exist.
    \def\br{\hspace*{\fill} \\* }
    % Math Jax compatability definitions
    \def\gt{>}
    \def\lt{<}
    % Document parameters
    \title{fourier\_analysis}
    
    
    

    % Pygments definitions
    
\makeatletter
\def\PY@reset{\let\PY@it=\relax \let\PY@bf=\relax%
    \let\PY@ul=\relax \let\PY@tc=\relax%
    \let\PY@bc=\relax \let\PY@ff=\relax}
\def\PY@tok#1{\csname PY@tok@#1\endcsname}
\def\PY@toks#1+{\ifx\relax#1\empty\else%
    \PY@tok{#1}\expandafter\PY@toks\fi}
\def\PY@do#1{\PY@bc{\PY@tc{\PY@ul{%
    \PY@it{\PY@bf{\PY@ff{#1}}}}}}}
\def\PY#1#2{\PY@reset\PY@toks#1+\relax+\PY@do{#2}}

\expandafter\def\csname PY@tok@w\endcsname{\def\PY@tc##1{\textcolor[rgb]{0.73,0.73,0.73}{##1}}}
\expandafter\def\csname PY@tok@c\endcsname{\let\PY@it=\textit\def\PY@tc##1{\textcolor[rgb]{0.25,0.50,0.50}{##1}}}
\expandafter\def\csname PY@tok@cp\endcsname{\def\PY@tc##1{\textcolor[rgb]{0.74,0.48,0.00}{##1}}}
\expandafter\def\csname PY@tok@k\endcsname{\let\PY@bf=\textbf\def\PY@tc##1{\textcolor[rgb]{0.00,0.50,0.00}{##1}}}
\expandafter\def\csname PY@tok@kp\endcsname{\def\PY@tc##1{\textcolor[rgb]{0.00,0.50,0.00}{##1}}}
\expandafter\def\csname PY@tok@kt\endcsname{\def\PY@tc##1{\textcolor[rgb]{0.69,0.00,0.25}{##1}}}
\expandafter\def\csname PY@tok@o\endcsname{\def\PY@tc##1{\textcolor[rgb]{0.40,0.40,0.40}{##1}}}
\expandafter\def\csname PY@tok@ow\endcsname{\let\PY@bf=\textbf\def\PY@tc##1{\textcolor[rgb]{0.67,0.13,1.00}{##1}}}
\expandafter\def\csname PY@tok@nb\endcsname{\def\PY@tc##1{\textcolor[rgb]{0.00,0.50,0.00}{##1}}}
\expandafter\def\csname PY@tok@nf\endcsname{\def\PY@tc##1{\textcolor[rgb]{0.00,0.00,1.00}{##1}}}
\expandafter\def\csname PY@tok@nc\endcsname{\let\PY@bf=\textbf\def\PY@tc##1{\textcolor[rgb]{0.00,0.00,1.00}{##1}}}
\expandafter\def\csname PY@tok@nn\endcsname{\let\PY@bf=\textbf\def\PY@tc##1{\textcolor[rgb]{0.00,0.00,1.00}{##1}}}
\expandafter\def\csname PY@tok@ne\endcsname{\let\PY@bf=\textbf\def\PY@tc##1{\textcolor[rgb]{0.82,0.25,0.23}{##1}}}
\expandafter\def\csname PY@tok@nv\endcsname{\def\PY@tc##1{\textcolor[rgb]{0.10,0.09,0.49}{##1}}}
\expandafter\def\csname PY@tok@no\endcsname{\def\PY@tc##1{\textcolor[rgb]{0.53,0.00,0.00}{##1}}}
\expandafter\def\csname PY@tok@nl\endcsname{\def\PY@tc##1{\textcolor[rgb]{0.63,0.63,0.00}{##1}}}
\expandafter\def\csname PY@tok@ni\endcsname{\let\PY@bf=\textbf\def\PY@tc##1{\textcolor[rgb]{0.60,0.60,0.60}{##1}}}
\expandafter\def\csname PY@tok@na\endcsname{\def\PY@tc##1{\textcolor[rgb]{0.49,0.56,0.16}{##1}}}
\expandafter\def\csname PY@tok@nt\endcsname{\let\PY@bf=\textbf\def\PY@tc##1{\textcolor[rgb]{0.00,0.50,0.00}{##1}}}
\expandafter\def\csname PY@tok@nd\endcsname{\def\PY@tc##1{\textcolor[rgb]{0.67,0.13,1.00}{##1}}}
\expandafter\def\csname PY@tok@s\endcsname{\def\PY@tc##1{\textcolor[rgb]{0.73,0.13,0.13}{##1}}}
\expandafter\def\csname PY@tok@sd\endcsname{\let\PY@it=\textit\def\PY@tc##1{\textcolor[rgb]{0.73,0.13,0.13}{##1}}}
\expandafter\def\csname PY@tok@si\endcsname{\let\PY@bf=\textbf\def\PY@tc##1{\textcolor[rgb]{0.73,0.40,0.53}{##1}}}
\expandafter\def\csname PY@tok@se\endcsname{\let\PY@bf=\textbf\def\PY@tc##1{\textcolor[rgb]{0.73,0.40,0.13}{##1}}}
\expandafter\def\csname PY@tok@sr\endcsname{\def\PY@tc##1{\textcolor[rgb]{0.73,0.40,0.53}{##1}}}
\expandafter\def\csname PY@tok@ss\endcsname{\def\PY@tc##1{\textcolor[rgb]{0.10,0.09,0.49}{##1}}}
\expandafter\def\csname PY@tok@sx\endcsname{\def\PY@tc##1{\textcolor[rgb]{0.00,0.50,0.00}{##1}}}
\expandafter\def\csname PY@tok@m\endcsname{\def\PY@tc##1{\textcolor[rgb]{0.40,0.40,0.40}{##1}}}
\expandafter\def\csname PY@tok@gh\endcsname{\let\PY@bf=\textbf\def\PY@tc##1{\textcolor[rgb]{0.00,0.00,0.50}{##1}}}
\expandafter\def\csname PY@tok@gu\endcsname{\let\PY@bf=\textbf\def\PY@tc##1{\textcolor[rgb]{0.50,0.00,0.50}{##1}}}
\expandafter\def\csname PY@tok@gd\endcsname{\def\PY@tc##1{\textcolor[rgb]{0.63,0.00,0.00}{##1}}}
\expandafter\def\csname PY@tok@gi\endcsname{\def\PY@tc##1{\textcolor[rgb]{0.00,0.63,0.00}{##1}}}
\expandafter\def\csname PY@tok@gr\endcsname{\def\PY@tc##1{\textcolor[rgb]{1.00,0.00,0.00}{##1}}}
\expandafter\def\csname PY@tok@ge\endcsname{\let\PY@it=\textit}
\expandafter\def\csname PY@tok@gs\endcsname{\let\PY@bf=\textbf}
\expandafter\def\csname PY@tok@gp\endcsname{\let\PY@bf=\textbf\def\PY@tc##1{\textcolor[rgb]{0.00,0.00,0.50}{##1}}}
\expandafter\def\csname PY@tok@go\endcsname{\def\PY@tc##1{\textcolor[rgb]{0.53,0.53,0.53}{##1}}}
\expandafter\def\csname PY@tok@gt\endcsname{\def\PY@tc##1{\textcolor[rgb]{0.00,0.27,0.87}{##1}}}
\expandafter\def\csname PY@tok@err\endcsname{\def\PY@bc##1{\setlength{\fboxsep}{0pt}\fcolorbox[rgb]{1.00,0.00,0.00}{1,1,1}{\strut ##1}}}
\expandafter\def\csname PY@tok@kc\endcsname{\let\PY@bf=\textbf\def\PY@tc##1{\textcolor[rgb]{0.00,0.50,0.00}{##1}}}
\expandafter\def\csname PY@tok@kd\endcsname{\let\PY@bf=\textbf\def\PY@tc##1{\textcolor[rgb]{0.00,0.50,0.00}{##1}}}
\expandafter\def\csname PY@tok@kn\endcsname{\let\PY@bf=\textbf\def\PY@tc##1{\textcolor[rgb]{0.00,0.50,0.00}{##1}}}
\expandafter\def\csname PY@tok@kr\endcsname{\let\PY@bf=\textbf\def\PY@tc##1{\textcolor[rgb]{0.00,0.50,0.00}{##1}}}
\expandafter\def\csname PY@tok@bp\endcsname{\def\PY@tc##1{\textcolor[rgb]{0.00,0.50,0.00}{##1}}}
\expandafter\def\csname PY@tok@fm\endcsname{\def\PY@tc##1{\textcolor[rgb]{0.00,0.00,1.00}{##1}}}
\expandafter\def\csname PY@tok@vc\endcsname{\def\PY@tc##1{\textcolor[rgb]{0.10,0.09,0.49}{##1}}}
\expandafter\def\csname PY@tok@vg\endcsname{\def\PY@tc##1{\textcolor[rgb]{0.10,0.09,0.49}{##1}}}
\expandafter\def\csname PY@tok@vi\endcsname{\def\PY@tc##1{\textcolor[rgb]{0.10,0.09,0.49}{##1}}}
\expandafter\def\csname PY@tok@vm\endcsname{\def\PY@tc##1{\textcolor[rgb]{0.10,0.09,0.49}{##1}}}
\expandafter\def\csname PY@tok@sa\endcsname{\def\PY@tc##1{\textcolor[rgb]{0.73,0.13,0.13}{##1}}}
\expandafter\def\csname PY@tok@sb\endcsname{\def\PY@tc##1{\textcolor[rgb]{0.73,0.13,0.13}{##1}}}
\expandafter\def\csname PY@tok@sc\endcsname{\def\PY@tc##1{\textcolor[rgb]{0.73,0.13,0.13}{##1}}}
\expandafter\def\csname PY@tok@dl\endcsname{\def\PY@tc##1{\textcolor[rgb]{0.73,0.13,0.13}{##1}}}
\expandafter\def\csname PY@tok@s2\endcsname{\def\PY@tc##1{\textcolor[rgb]{0.73,0.13,0.13}{##1}}}
\expandafter\def\csname PY@tok@sh\endcsname{\def\PY@tc##1{\textcolor[rgb]{0.73,0.13,0.13}{##1}}}
\expandafter\def\csname PY@tok@s1\endcsname{\def\PY@tc##1{\textcolor[rgb]{0.73,0.13,0.13}{##1}}}
\expandafter\def\csname PY@tok@mb\endcsname{\def\PY@tc##1{\textcolor[rgb]{0.40,0.40,0.40}{##1}}}
\expandafter\def\csname PY@tok@mf\endcsname{\def\PY@tc##1{\textcolor[rgb]{0.40,0.40,0.40}{##1}}}
\expandafter\def\csname PY@tok@mh\endcsname{\def\PY@tc##1{\textcolor[rgb]{0.40,0.40,0.40}{##1}}}
\expandafter\def\csname PY@tok@mi\endcsname{\def\PY@tc##1{\textcolor[rgb]{0.40,0.40,0.40}{##1}}}
\expandafter\def\csname PY@tok@il\endcsname{\def\PY@tc##1{\textcolor[rgb]{0.40,0.40,0.40}{##1}}}
\expandafter\def\csname PY@tok@mo\endcsname{\def\PY@tc##1{\textcolor[rgb]{0.40,0.40,0.40}{##1}}}
\expandafter\def\csname PY@tok@ch\endcsname{\let\PY@it=\textit\def\PY@tc##1{\textcolor[rgb]{0.25,0.50,0.50}{##1}}}
\expandafter\def\csname PY@tok@cm\endcsname{\let\PY@it=\textit\def\PY@tc##1{\textcolor[rgb]{0.25,0.50,0.50}{##1}}}
\expandafter\def\csname PY@tok@cpf\endcsname{\let\PY@it=\textit\def\PY@tc##1{\textcolor[rgb]{0.25,0.50,0.50}{##1}}}
\expandafter\def\csname PY@tok@c1\endcsname{\let\PY@it=\textit\def\PY@tc##1{\textcolor[rgb]{0.25,0.50,0.50}{##1}}}
\expandafter\def\csname PY@tok@cs\endcsname{\let\PY@it=\textit\def\PY@tc##1{\textcolor[rgb]{0.25,0.50,0.50}{##1}}}

\def\PYZbs{\char`\\}
\def\PYZus{\char`\_}
\def\PYZob{\char`\{}
\def\PYZcb{\char`\}}
\def\PYZca{\char`\^}
\def\PYZam{\char`\&}
\def\PYZlt{\char`\<}
\def\PYZgt{\char`\>}
\def\PYZsh{\char`\#}
\def\PYZpc{\char`\%}
\def\PYZdl{\char`\$}
\def\PYZhy{\char`\-}
\def\PYZsq{\char`\'}
\def\PYZdq{\char`\"}
\def\PYZti{\char`\~}
% for compatibility with earlier versions
\def\PYZat{@}
\def\PYZlb{[}
\def\PYZrb{]}
\makeatother


    % Exact colors from NB
    \definecolor{incolor}{rgb}{0.0, 0.0, 0.5}
    \definecolor{outcolor}{rgb}{0.545, 0.0, 0.0}



    
    % Prevent overflowing lines due to hard-to-break entities
    \sloppy 
    % Setup hyperref package
    \hypersetup{
      breaklinks=true,  % so long urls are correctly broken across lines
      colorlinks=true,
      urlcolor=urlcolor,
      linkcolor=linkcolor,
      citecolor=citecolor,
      }
    % Slightly bigger margins than the latex defaults
    
    \geometry{verbose,tmargin=1in,bmargin=1in,lmargin=1in,rmargin=1in}
    
    

    \begin{document}
    
    
    \maketitle
    
    

    
    \subsection{Setup Simulation}\label{setup-simulation}

Import needed libraries and define constants

    \begin{Verbatim}[commandchars=\\\{\}]
{\color{incolor}In [{\color{incolor}1}]:} \PY{c+c1}{\PYZsh{} Imports}
        \PY{k+kn}{from} \PY{n+nn}{rcfdtd\PYZus{}sim} \PY{k}{import} \PY{n}{Sim}\PY{p}{,} \PY{n}{Current}\PY{p}{,} \PY{n}{Mat}\PY{p}{,} \PY{n}{vis}
        \PY{k+kn}{import} \PY{n+nn}{numpy} \PY{k}{as} \PY{n+nn}{np}
        \PY{k+kn}{from} \PY{n+nn}{scipy}\PY{n+nn}{.}\PY{n+nn}{fftpack} \PY{k}{import} \PY{n}{fft}
        \PY{k+kn}{from} \PY{n+nn}{matplotlib} \PY{k}{import} \PY{n}{pyplot} \PY{k}{as} \PY{n}{plt}
        \PY{k+kn}{from} \PY{n+nn}{pathlib} \PY{k}{import} \PY{n}{Path}
        \PY{c+c1}{\PYZsh{} Determine file save name}
        \PY{n}{fsave} \PY{o}{=} \PY{l+s+s1}{\PYZsq{}}\PY{l+s+s1}{dat1\PYZus{}1ps.npz}\PY{l+s+s1}{\PYZsq{}}
        \PY{c+c1}{\PYZsh{} Constants}
        \PY{n}{c0} \PY{o}{=} \PY{l+m+mi}{300} \PY{c+c1}{\PYZsh{} um/ps}
        \PY{n}{di} \PY{o}{=} \PY{l+m+mf}{0.003} \PY{c+c1}{\PYZsh{} 0.003 um}
        \PY{n}{dn} \PY{o}{=} \PY{n}{di}\PY{o}{/}\PY{n}{c0} \PY{c+c1}{\PYZsh{} (0.003 um) / (300 um/ps) = 0.00001 ps = 0.01 fs}
        \PY{n}{epsilon0} \PY{o}{=} \PY{l+m+mf}{8.85419e18} \PY{c+c1}{\PYZsh{} (A\PYZca{}2 ps\PYZca{}4)/(kg um\PYZca{}3)}
        \PY{n}{mu0} \PY{o}{=} \PY{l+m+mf}{1.25664e\PYZhy{}24} \PY{c+c1}{\PYZsh{} (kg um)/(A\PYZca{}2 ps\PYZca{}2)}
\end{Verbatim}


    Define simulation bounds and calculate length in indicies

    \begin{Verbatim}[commandchars=\\\{\}]
{\color{incolor}In [{\color{incolor}2}]:} \PY{c+c1}{\PYZsh{} Define bounds}
        \PY{n}{i0} \PY{o}{=} \PY{o}{\PYZhy{}}\PY{l+m+mi}{3} \PY{c+c1}{\PYZsh{} \PYZhy{}3um}
        \PY{n}{i1} \PY{o}{=} \PY{l+m+mi}{6} \PY{c+c1}{\PYZsh{} 6 um}
        \PY{n}{n0} \PY{o}{=} \PY{o}{\PYZhy{}}\PY{l+m+mf}{1e\PYZhy{}2} \PY{c+c1}{\PYZsh{} \PYZhy{}1e\PYZhy{}3 ps = \PYZhy{}10 fs}
        \PY{n}{n1} \PY{o}{=} \PY{l+m+mi}{1} \PY{c+c1}{\PYZsh{} 1 ps = 1000 fs}
        \PY{c+c1}{\PYZsh{} Calculate dimensions}
        \PY{n}{nlen}\PY{p}{,} \PY{n}{ilen} \PY{o}{=} \PY{n}{Sim}\PY{o}{.}\PY{n}{calc\PYZus{}dims}\PY{p}{(}\PY{n}{n0}\PY{p}{,} \PY{n}{n1}\PY{p}{,} \PY{n}{dn}\PY{p}{,} \PY{n}{i0}\PY{p}{,} \PY{n}{i1}\PY{p}{,} \PY{n}{di}\PY{p}{)}
\end{Verbatim}


    Create our time and space arrays to help construct our material and
current pulse

    \begin{Verbatim}[commandchars=\\\{\}]
{\color{incolor}In [{\color{incolor}3}]:} \PY{c+c1}{\PYZsh{} Create a arrays that hold the value of the center of each cell}
        \PY{n}{t} \PY{o}{=} \PY{n}{np}\PY{o}{.}\PY{n}{linspace}\PY{p}{(}\PY{n}{n0}\PY{o}{+}\PY{n}{dn}\PY{o}{/}\PY{l+m+mi}{2}\PY{p}{,} \PY{n}{n1}\PY{o}{+}\PY{n}{dn}\PY{o}{/}\PY{l+m+mi}{2}\PY{p}{,} \PY{n}{nlen}\PY{p}{,} \PY{n}{endpoint}\PY{o}{=}\PY{k+kc}{False}\PY{p}{)}
        \PY{n}{z} \PY{o}{=} \PY{n}{np}\PY{o}{.}\PY{n}{linspace}\PY{p}{(}\PY{n}{i0}\PY{o}{+}\PY{n}{di}\PY{o}{/}\PY{l+m+mi}{2}\PY{p}{,} \PY{n}{i1}\PY{o}{+}\PY{n}{di}\PY{o}{/}\PY{l+m+mi}{2}\PY{p}{,} \PY{n}{ilen}\PY{p}{,} \PY{n}{endpoint}\PY{o}{=}\PY{k+kc}{False}\PY{p}{)}
\end{Verbatim}


    \subsection{Setup Current}\label{setup-current}

Specify the location of our current pulse in time and space

    \begin{Verbatim}[commandchars=\\\{\}]
{\color{incolor}In [{\color{incolor}4}]:} \PY{n}{cp\PYZus{}loc\PYZus{}val} \PY{o}{=} \PY{o}{\PYZhy{}}\PY{l+m+mi}{2}
        \PY{n}{cp\PYZus{}time\PYZus{}val} \PY{o}{=} \PY{l+m+mi}{0}
\end{Verbatim}


    Determine the simulation indicies that correspond to these locations

    \begin{Verbatim}[commandchars=\\\{\}]
{\color{incolor}In [{\color{incolor}5}]:} \PY{c+c1}{\PYZsh{} Find indicies}
        \PY{n}{cp\PYZus{}loc\PYZus{}ind} \PY{o}{=} \PY{n}{np}\PY{o}{.}\PY{n}{argmin}\PY{p}{(}\PY{n}{np}\PY{o}{.}\PY{n}{abs}\PY{p}{(}\PY{n}{np}\PY{o}{.}\PY{n}{subtract}\PY{p}{(}\PY{n}{z}\PY{p}{,} \PY{n}{cp\PYZus{}loc\PYZus{}val}\PY{p}{)}\PY{p}{)}\PY{p}{)}
        \PY{n}{cp\PYZus{}time\PYZus{}ind} \PY{o}{=} \PY{n}{np}\PY{o}{.}\PY{n}{argmin}\PY{p}{(}\PY{n}{np}\PY{o}{.}\PY{n}{abs}\PY{p}{(}\PY{n}{np}\PY{o}{.}\PY{n}{subtract}\PY{p}{(}\PY{n}{t}\PY{p}{,} \PY{n}{cp\PYZus{}time\PYZus{}val}\PY{p}{)}\PY{p}{)}\PY{p}{)}
        \PY{c+c1}{\PYZsh{} Find start and end indicies in time}
        \PY{n}{cp\PYZus{}time\PYZus{}s} \PY{o}{=} \PY{n}{cp\PYZus{}time\PYZus{}ind} \PY{o}{\PYZhy{}} \PY{l+m+mi}{350}
        \PY{n}{cp\PYZus{}time\PYZus{}e} \PY{o}{=} \PY{n}{cp\PYZus{}time\PYZus{}ind} \PY{o}{+} \PY{l+m+mi}{350}
\end{Verbatim}


    Create the current pulse

    \begin{Verbatim}[commandchars=\\\{\}]
{\color{incolor}In [{\color{incolor}6}]:} \PY{c+c1}{\PYZsh{} Make pulse}
        \PY{n}{cpulse} \PY{o}{=} \PY{n}{np}\PY{o}{.}\PY{n}{append}\PY{p}{(}\PY{n}{np}\PY{o}{.}\PY{n}{diff}\PY{p}{(}\PY{n}{np}\PY{o}{.}\PY{n}{diff}\PY{p}{(}\PY{n}{np}\PY{o}{.}\PY{n}{exp}\PY{p}{(}\PY{o}{\PYZhy{}}\PY{p}{(}\PY{p}{(}\PY{n}{t}\PY{p}{[}\PY{n}{cp\PYZus{}time\PYZus{}s}\PY{p}{:}\PY{n}{cp\PYZus{}time\PYZus{}e}\PY{p}{]}\PY{o}{\PYZhy{}}\PY{n}{cp\PYZus{}time\PYZus{}val}\PY{p}{)}\PY{o}{*}\PY{o}{*}\PY{l+m+mi}{2}\PY{p}{)}\PY{o}{/}\PY{p}{(}\PY{l+m+mf}{1e\PYZhy{}6}\PY{p}{)}\PY{p}{)}\PY{p}{)}\PY{p}{)}\PY{p}{,} \PY{p}{[}\PY{l+m+mi}{0}\PY{p}{,}\PY{l+m+mi}{0}\PY{p}{]}\PY{p}{)}
        \PY{c+c1}{\PYZsh{} Plot}
        \PY{n}{plt}\PY{o}{.}\PY{n}{plot}\PY{p}{(}\PY{l+m+mf}{1e3}\PY{o}{*}\PY{n}{t}\PY{p}{[}\PY{n}{cp\PYZus{}time\PYZus{}s}\PY{p}{:}\PY{n}{cp\PYZus{}time\PYZus{}e}\PY{p}{]}\PY{p}{,} \PY{l+m+mf}{1e\PYZhy{}3}\PY{o}{*}\PY{n}{cpulse}\PY{p}{)} \PY{c+c1}{\PYZsh{} 1e3 scale factor converts from ps \PYZhy{}\PYZgt{} fs, and 1e\PYZhy{}3 scale factor converts from A \PYZhy{}\PYZgt{} mA}
        \PY{n}{plt}\PY{o}{.}\PY{n}{xlabel}\PY{p}{(}\PY{l+s+s1}{\PYZsq{}}\PY{l+s+s1}{time [fs]}\PY{l+s+s1}{\PYZsq{}}\PY{p}{)}
        \PY{n}{plt}\PY{o}{.}\PY{n}{ylabel}\PY{p}{(}\PY{l+s+s1}{\PYZsq{}}\PY{l+s+s1}{current [mA]}\PY{l+s+s1}{\PYZsq{}}\PY{p}{)}
        \PY{n}{plt}\PY{o}{.}\PY{n}{show}\PY{p}{(}\PY{p}{)}
        \PY{c+c1}{\PYZsh{} Create Current object}
        \PY{n}{current} \PY{o}{=} \PY{n}{Current}\PY{p}{(}\PY{n}{nlen}\PY{p}{,} \PY{n}{ilen}\PY{p}{,} \PY{n}{cp\PYZus{}time\PYZus{}s}\PY{p}{,} \PY{n}{cp\PYZus{}loc\PYZus{}ind}\PY{p}{,} \PY{n}{cpulse}\PY{p}{)}
\end{Verbatim}


    \begin{center}
    \adjustimage{max size={0.9\linewidth}{0.9\paperheight}}{output_11_0.png}
    \end{center}
    { \hspace*{\fill} \\}
    
    \subsection{Setup Material}\label{setup-material}

Specify the location of our material (which will be \(4\)um in length)

    \begin{Verbatim}[commandchars=\\\{\}]
{\color{incolor}In [{\color{incolor}7}]:} \PY{c+c1}{\PYZsh{} Set material length}
        \PY{n}{m\PYZus{}len} \PY{o}{=} \PY{l+m+mi}{4} \PY{c+c1}{\PYZsh{} 4 um}
        \PY{c+c1}{\PYZsh{} Set locations}
        \PY{n}{m\PYZus{}s\PYZus{}val} \PY{o}{=} \PY{l+m+mi}{0}
        \PY{n}{m\PYZus{}e\PYZus{}val} \PY{o}{=} \PY{n}{m\PYZus{}s\PYZus{}val} \PY{o}{+} \PY{n}{m\PYZus{}len}
\end{Verbatim}


    Calculate the starting and ending indicies of our material

    \begin{Verbatim}[commandchars=\\\{\}]
{\color{incolor}In [{\color{incolor}8}]:} \PY{n}{m\PYZus{}s\PYZus{}ind} \PY{o}{=} \PY{n}{np}\PY{o}{.}\PY{n}{argmin}\PY{p}{(}\PY{n}{np}\PY{o}{.}\PY{n}{abs}\PY{p}{(}\PY{n}{np}\PY{o}{.}\PY{n}{subtract}\PY{p}{(}\PY{n}{z}\PY{p}{,} \PY{n}{m\PYZus{}s\PYZus{}val}\PY{p}{)}\PY{p}{)}\PY{p}{)}
        \PY{n}{m\PYZus{}e\PYZus{}ind} \PY{o}{=} \PY{n}{np}\PY{o}{.}\PY{n}{argmin}\PY{p}{(}\PY{n}{np}\PY{o}{.}\PY{n}{abs}\PY{p}{(}\PY{n}{np}\PY{o}{.}\PY{n}{subtract}\PY{p}{(}\PY{n}{z}\PY{p}{,} \PY{n}{m\PYZus{}e\PYZus{}val}\PY{p}{)}\PY{p}{)}\PY{p}{)}
\end{Verbatim}


    Setup material behavior

    \begin{Verbatim}[commandchars=\\\{\}]
{\color{incolor}In [{\color{incolor}9}]:} \PY{c+c1}{\PYZsh{} Set constants}
        \PY{n}{gamma} \PY{o}{=} \PY{n}{np}\PY{o}{.}\PY{n}{complex64}\PY{p}{(}\PY{l+m+mf}{0.01} \PY{o}{*} \PY{l+m+mi}{2} \PY{o}{*} \PY{n}{np}\PY{o}{.}\PY{n}{pi}\PY{p}{)}
        \PY{n}{omega} \PY{o}{=} \PY{n}{np}\PY{o}{.}\PY{n}{complex64}\PY{p}{(}\PY{l+m+mf}{0.0}\PY{p}{)}
        \PY{n}{a1} \PY{o}{=} \PY{n}{np}\PY{o}{.}\PY{n}{complex64}\PY{p}{(}\PY{l+m+mi}{0}\PY{p}{)}
        \PY{n}{a2} \PY{o}{=} \PY{n}{np}\PY{o}{.}\PY{n}{complex64}\PY{p}{(}\PY{l+m+mi}{0}\PY{p}{)}
        \PY{c+c1}{\PYZsh{} Calculate beta}
        \PY{n}{beta} \PY{o}{=} \PY{n}{np}\PY{o}{.}\PY{n}{sqrt}\PY{p}{(}\PY{n}{np}\PY{o}{.}\PY{n}{add}\PY{p}{(}\PY{n}{np}\PY{o}{.}\PY{n}{square}\PY{p}{(}\PY{n}{gamma}\PY{p}{)}\PY{p}{,} \PY{o}{\PYZhy{}}\PY{n}{np}\PY{o}{.}\PY{n}{square}\PY{p}{(}\PY{n}{omega}\PY{p}{)}\PY{p}{)}\PY{p}{)}
\end{Verbatim}


    Create our material behavior matrices

    \begin{Verbatim}[commandchars=\\\{\}]
{\color{incolor}In [{\color{incolor}10}]:} \PY{c+c1}{\PYZsh{} Determine matrix length}
         \PY{n}{mlen} \PY{o}{=} \PY{n}{m\PYZus{}e\PYZus{}ind} \PY{o}{\PYZhy{}} \PY{n}{m\PYZus{}s\PYZus{}ind}
         \PY{c+c1}{\PYZsh{} Create matrices}
         \PY{n}{m} \PY{o}{=} \PY{n}{np}\PY{o}{.}\PY{n}{ones}\PY{p}{(}\PY{p}{(}\PY{l+m+mi}{1}\PY{p}{,} \PY{n}{mlen}\PY{p}{)}\PY{p}{,} \PY{n}{dtype}\PY{o}{=}\PY{n}{np}\PY{o}{.}\PY{n}{complex64}\PY{p}{)}
         \PY{n}{mgamma} \PY{o}{=} \PY{n}{m} \PY{o}{*} \PY{n}{gamma}
         \PY{n}{mbeta} \PY{o}{=} \PY{n}{m} \PY{o}{*} \PY{n}{beta}
         \PY{n}{ma1} \PY{o}{=} \PY{n}{m} \PY{o}{*} \PY{n}{a1}
         \PY{n}{ma2} \PY{o}{=} \PY{n}{m} \PY{o}{*} \PY{n}{a2}
\end{Verbatim}


    Create our material object

    \begin{Verbatim}[commandchars=\\\{\}]
{\color{incolor}In [{\color{incolor}11}]:} \PY{n}{inf\PYZus{}perm} \PY{o}{=} \PY{l+m+mi}{8}
         \PY{n}{material} \PY{o}{=} \PY{n}{Mat}\PY{p}{(}\PY{n}{dn}\PY{p}{,} \PY{n}{ilen}\PY{p}{,} \PY{n}{m\PYZus{}s\PYZus{}ind}\PY{p}{,} \PY{n}{inf\PYZus{}perm}\PY{p}{,} \PY{n}{ma1}\PY{p}{,} \PY{n}{ma2}\PY{p}{,} \PY{n}{mgamma}\PY{p}{,} \PY{n}{mbeta}\PY{p}{)}
\end{Verbatim}


    \subsection{Running the Simulation}\label{running-the-simulation}

Create and run our simulation (or load simulation if one already exists)

    \begin{Verbatim}[commandchars=\\\{\}]
{\color{incolor}In [{\color{incolor}12}]:} \PY{c+c1}{\PYZsh{} Create Sim object}
         \PY{n}{s} \PY{o}{=} \PY{n}{Sim}\PY{p}{(}\PY{n}{i0}\PY{p}{,} \PY{n}{i1}\PY{p}{,} \PY{n}{di}\PY{p}{,} \PY{n}{n0}\PY{p}{,} \PY{n}{n1}\PY{p}{,} \PY{n}{dn}\PY{p}{,} \PY{n}{epsilon0}\PY{p}{,} \PY{n}{mu0}\PY{p}{,} \PY{l+s+s1}{\PYZsq{}}\PY{l+s+s1}{absorbing}\PY{l+s+s1}{\PYZsq{}}\PY{p}{,} \PY{n}{current}\PY{p}{,} \PY{n}{material}\PY{p}{,} \PY{n}{nstore}\PY{o}{=}\PY{n+nb}{int}\PY{p}{(}\PY{n}{nlen}\PY{o}{/}\PY{l+m+mi}{20}\PY{p}{)}\PY{p}{,} \PY{n}{storelocs}\PY{o}{=}\PY{p}{[}\PY{l+m+mi}{1}\PY{p}{,}\PY{n}{ilen}\PY{o}{\PYZhy{}}\PY{l+m+mi}{2}\PY{p}{]}\PY{p}{)}
         \PY{c+c1}{\PYZsh{} Run simulation if simulation save doesn\PYZsq{}t exist}
         \PY{n}{sim\PYZus{}file} \PY{o}{=} \PY{n}{Path}\PY{p}{(}\PY{n}{fsave}\PY{p}{)}
         \PY{k}{if} \PY{n}{sim\PYZus{}file}\PY{o}{.}\PY{n}{is\PYZus{}file}\PY{p}{(}\PY{p}{)}\PY{p}{:}
             \PY{c+c1}{\PYZsh{} Load results}
             \PY{n}{dat} \PY{o}{=} \PY{n}{np}\PY{o}{.}\PY{n}{load}\PY{p}{(}\PY{n}{fsave}\PY{p}{)}
             \PY{n}{n} \PY{o}{=} \PY{n}{dat}\PY{p}{[}\PY{l+s+s1}{\PYZsq{}}\PY{l+s+s1}{n}\PY{l+s+s1}{\PYZsq{}}\PY{p}{]}
             \PY{n}{ls} \PY{o}{=} \PY{n}{dat}\PY{p}{[}\PY{l+s+s1}{\PYZsq{}}\PY{l+s+s1}{ls}\PY{l+s+s1}{\PYZsq{}}\PY{p}{]}
             \PY{n}{els} \PY{o}{=} \PY{n}{dat}\PY{p}{[}\PY{l+s+s1}{\PYZsq{}}\PY{l+s+s1}{els}\PY{l+s+s1}{\PYZsq{}}\PY{p}{]}
             \PY{n}{erls} \PY{o}{=} \PY{n}{dat}\PY{p}{[}\PY{l+s+s1}{\PYZsq{}}\PY{l+s+s1}{erls}\PY{l+s+s1}{\PYZsq{}}\PY{p}{]}
             \PY{n}{hls} \PY{o}{=} \PY{n}{dat}\PY{p}{[}\PY{l+s+s1}{\PYZsq{}}\PY{l+s+s1}{hls}\PY{l+s+s1}{\PYZsq{}}\PY{p}{]}
             \PY{n}{hrls} \PY{o}{=} \PY{n}{dat}\PY{p}{[}\PY{l+s+s1}{\PYZsq{}}\PY{l+s+s1}{hrls}\PY{l+s+s1}{\PYZsq{}}\PY{p}{]}
         \PY{k}{else}\PY{p}{:}
             \PY{c+c1}{\PYZsh{} Run simulation}
             \PY{n}{s}\PY{o}{.}\PY{n}{simulate}\PY{p}{(}\PY{p}{)}
             \PY{c+c1}{\PYZsh{} Export and save arrays}
             \PY{n}{n}\PY{p}{,} \PY{n}{ls}\PY{p}{,} \PY{n}{els}\PY{p}{,} \PY{n}{erls}\PY{p}{,} \PY{n}{hls}\PY{p}{,} \PY{n}{hrls} \PY{o}{=} \PY{n}{s}\PY{o}{.}\PY{n}{export\PYZus{}locs}\PY{p}{(}\PY{p}{)}
             \PY{n}{np}\PY{o}{.}\PY{n}{savez}\PY{p}{(}\PY{n}{fsave}\PY{p}{,} \PY{n}{n}\PY{o}{=}\PY{n}{n}\PY{p}{,} \PY{n}{ls}\PY{o}{=}\PY{n}{ls}\PY{p}{,} \PY{n}{els}\PY{o}{=}\PY{n}{els}\PY{p}{,} \PY{n}{erls}\PY{o}{=}\PY{n}{erls}\PY{p}{,} \PY{n}{hls}\PY{o}{=}\PY{n}{hls}\PY{p}{,} \PY{n}{hrls}\PY{o}{=}\PY{n}{hrls}\PY{p}{)}
\end{Verbatim}


    \subsection{Finding the Transmission and Reflection
Coefficients}\label{finding-the-transmission-and-reflection-coefficients}

We ultimately wish to calculate the reflection and transmission
coefficients
\[R(\omega)=\frac{E_r(\omega)}{E_i(\omega)} \qquad T(\omega)=\frac{E_t(\omega)}{E_i(\omega)}\]
which means that we need to extract our incident E-field \(E_i\),
reflected E-field \(E_r\), and transmitted E-field \(E_t\) from our
E-field and reference E-field passing through \(i=1\) and
\(i=\text{ilen}-2\) (these fields are contained in \texttt{els} and
\texttt{erls}). The incident E-field \(E_i\) should be equal to the
reference E-field passing through \(i=\text{ilen}-2\). Since the current
is only non-zero from \(-3\)fs to \(3\)fs and since it will take light
propegating at \(300\)um/ps \(26.6\)fs to reach the end of the
simulation at \(6\)um from \(-2\)um, we plot the reference E-field from
\(23.6\)fs to \(29.6\)fs below

    \begin{Verbatim}[commandchars=\\\{\}]
{\color{incolor}In [{\color{incolor}13}]:} \PY{c+c1}{\PYZsh{} Plot the E\PYZhy{}field passing through i=ilen\PYZhy{}2}
         \PY{n}{inc} \PY{o}{=} \PY{n}{erls}\PY{p}{[}\PY{p}{:}\PY{p}{,}\PY{l+m+mi}{1}\PY{p}{]} \PY{c+c1}{\PYZsh{} Extract the transmitted reference E\PYZhy{}field as the incident field}
         \PY{n}{plt}\PY{o}{.}\PY{n}{plot}\PY{p}{(}\PY{l+m+mf}{1e3}\PY{o}{*}\PY{n}{n}\PY{p}{,} \PY{n}{np}\PY{o}{.}\PY{n}{real}\PY{p}{(}\PY{n}{inc}\PY{p}{)}\PY{p}{)} \PY{c+c1}{\PYZsh{} 1e3 scale factor converts from ps \PYZhy{}\PYZgt{} fs}
         \PY{n}{plt}\PY{o}{.}\PY{n}{ylabel}\PY{p}{(}\PY{l+s+s1}{\PYZsq{}}\PY{l+s+s1}{current [?]}\PY{l+s+s1}{\PYZsq{}}\PY{p}{)}
         \PY{n}{plt}\PY{o}{.}\PY{n}{xlabel}\PY{p}{(}\PY{l+s+s1}{\PYZsq{}}\PY{l+s+s1}{time [fs]}\PY{l+s+s1}{\PYZsq{}}\PY{p}{)}
         \PY{c+c1}{\PYZsh{} Lets zoom in on the x\PYZhy{}axis as described above}
         \PY{n}{plt}\PY{o}{.}\PY{n}{xlim}\PY{p}{(}\PY{l+m+mf}{23.6}\PY{p}{,}\PY{l+m+mf}{29.6}\PY{p}{)}
         \PY{n}{plt}\PY{o}{.}\PY{n}{show}\PY{p}{(}\PY{p}{)}
\end{Verbatim}


    \begin{center}
    \adjustimage{max size={0.9\linewidth}{0.9\paperheight}}{output_25_0.png}
    \end{center}
    { \hspace*{\fill} \\}
    
    The pulse is (roughly) centered as expected. We can extract
\(E_i(\omega)\) from \(E^{i=\text{ilen}-2}_\text{ref}(t)\) via
\[\mathcal{F}\left\{E^{i=\text{ilen}-2}_\text{ref}(t)\right\}=E_i(\omega)\]
where \(\mathcal{F}\left\{A(t)\right\}=A(\omega)\) denotes a Fourier
Transform on \(A\)

    \begin{Verbatim}[commandchars=\\\{\}]
{\color{incolor}In [{\color{incolor}14}]:} \PY{c+c1}{\PYZsh{} Perform FFT}
         \PY{n}{incf} \PY{o}{=} \PY{n}{fft}\PY{p}{(}\PY{n}{inc}\PY{p}{)}
         \PY{c+c1}{\PYZsh{} Discard negative frequencies (at nlen/2 and above), keep DC (at 0)}
         \PY{n}{incf} \PY{o}{=} \PY{n}{incf}\PY{p}{[}\PY{l+m+mi}{0}\PY{p}{:}\PY{n+nb}{int}\PY{p}{(}\PY{n}{nlen}\PY{o}{/}\PY{l+m+mi}{2}\PY{p}{)}\PY{p}{]}
         \PY{c+c1}{\PYZsh{} Determine the frequency that each value in eef corresponds to}
         \PY{n}{nf} \PY{o}{=} \PY{n}{np}\PY{o}{.}\PY{n}{linspace}\PY{p}{(}\PY{l+m+mi}{0}\PY{p}{,} \PY{l+m+mi}{1}\PY{o}{/}\PY{p}{(}\PY{l+m+mi}{2}\PY{o}{*}\PY{n}{dn}\PY{p}{)}\PY{p}{,} \PY{n+nb}{int}\PY{p}{(}\PY{n}{nlen}\PY{o}{/}\PY{l+m+mi}{2}\PY{p}{)}\PY{p}{)}
\end{Verbatim}


    \begin{Verbatim}[commandchars=\\\{\}]
{\color{incolor}In [{\color{incolor}15}]:} \PY{c+c1}{\PYZsh{} Plot}
         \PY{n}{fig}\PY{p}{,} \PY{p}{(}\PY{n}{ax0}\PY{p}{,} \PY{n}{ax1}\PY{p}{)} \PY{o}{=} \PY{n}{plt}\PY{o}{.}\PY{n}{subplots}\PY{p}{(}\PY{n}{nrows}\PY{o}{=}\PY{l+m+mi}{2}\PY{p}{,} \PY{n}{sharex}\PY{o}{=}\PY{k+kc}{True}\PY{p}{)}
         \PY{n}{ax0}\PY{o}{.}\PY{n}{plot}\PY{p}{(}\PY{n}{nf}\PY{p}{,} \PY{n}{np}\PY{o}{.}\PY{n}{absolute}\PY{p}{(}\PY{n}{incf}\PY{p}{)}\PY{p}{)}
         \PY{n}{ax0}\PY{o}{.}\PY{n}{set\PYZus{}ylabel}\PY{p}{(}\PY{l+s+s1}{\PYZsq{}}\PY{l+s+s1}{magnitude [?]}\PY{l+s+s1}{\PYZsq{}}\PY{p}{)}
         \PY{n}{ax1}\PY{o}{.}\PY{n}{plot}\PY{p}{(}\PY{n}{nf}\PY{p}{,} \PY{n}{np}\PY{o}{.}\PY{n}{angle}\PY{p}{(}\PY{n}{incf}\PY{p}{)}\PY{p}{)}
         \PY{n}{ax1}\PY{o}{.}\PY{n}{set\PYZus{}ylabel}\PY{p}{(}\PY{l+s+s1}{\PYZsq{}}\PY{l+s+s1}{phase [rad]}\PY{l+s+s1}{\PYZsq{}}\PY{p}{)}
         \PY{n}{plt}\PY{o}{.}\PY{n}{xlabel}\PY{p}{(}\PY{l+s+s1}{\PYZsq{}}\PY{l+s+s1}{frequency [THz]}\PY{l+s+s1}{\PYZsq{}}\PY{p}{)}
         \PY{n}{plt}\PY{o}{.}\PY{n}{xlim}\PY{p}{(}\PY{l+m+mi}{0}\PY{p}{,}\PY{l+m+mi}{1000}\PY{p}{)} \PY{c+c1}{\PYZsh{} Truncate at 1THz}
         \PY{n}{plt}\PY{o}{.}\PY{n}{show}\PY{p}{(}\PY{p}{)}
\end{Verbatim}


    \begin{center}
    \adjustimage{max size={0.9\linewidth}{0.9\paperheight}}{output_28_0.png}
    \end{center}
    { \hspace*{\fill} \\}
    
    We can determine the transmitted E-field \(E_t\) from the E-field
passing through \(i=\text{ilen}-2\). We plot this field below

    \begin{Verbatim}[commandchars=\\\{\}]
{\color{incolor}In [{\color{incolor}16}]:} \PY{c+c1}{\PYZsh{} Plot the E\PYZhy{}field passing through i=ilen\PYZhy{}2}
         \PY{n}{trans} \PY{o}{=} \PY{n}{els}\PY{p}{[}\PY{p}{:}\PY{p}{,}\PY{l+m+mi}{1}\PY{p}{]} \PY{c+c1}{\PYZsh{} Extract the E\PYZhy{}field as the transmitted field}
         \PY{n}{plt}\PY{o}{.}\PY{n}{plot}\PY{p}{(}\PY{l+m+mf}{1e3}\PY{o}{*}\PY{n}{n}\PY{p}{,} \PY{n}{np}\PY{o}{.}\PY{n}{real}\PY{p}{(}\PY{n}{trans}\PY{p}{)}\PY{p}{)} \PY{c+c1}{\PYZsh{} 1e3 scale factor converts from ps \PYZhy{}\PYZgt{} fs}
         \PY{n}{plt}\PY{o}{.}\PY{n}{ylabel}\PY{p}{(}\PY{l+s+s1}{\PYZsq{}}\PY{l+s+s1}{real magnitude [?]}\PY{l+s+s1}{\PYZsq{}}\PY{p}{)}
         \PY{n}{plt}\PY{o}{.}\PY{n}{xlabel}\PY{p}{(}\PY{l+s+s1}{\PYZsq{}}\PY{l+s+s1}{time [fs]}\PY{l+s+s1}{\PYZsq{}}\PY{p}{)}
         \PY{n}{plt}\PY{o}{.}\PY{n}{show}\PY{p}{(}\PY{p}{)}
\end{Verbatim}


    \begin{center}
    \adjustimage{max size={0.9\linewidth}{0.9\paperheight}}{output_30_0.png}
    \end{center}
    { \hspace*{\fill} \\}
    
    Since there will be an infinite number of reflections inside of the
material, we will never capture the entire transmitted E-field. However,
if we run the simulation for a long enough time the E-field remaining in
the material will become negligable. We extract \(E_t(\omega)\) via
\[\mathcal{F}\left\{E^{i=\text{ilen}-2}(t)\right\}=E_t(\omega)\] where
again \(\mathcal{F}\left\{A(t)\right\}=A(\omega)\) denotes a Fourier
Transform on \(A\)

    \begin{Verbatim}[commandchars=\\\{\}]
{\color{incolor}In [{\color{incolor}17}]:} \PY{c+c1}{\PYZsh{} Perform FFT}
         \PY{n}{transf} \PY{o}{=} \PY{n}{fft}\PY{p}{(}\PY{n}{trans}\PY{p}{)}
         \PY{c+c1}{\PYZsh{} Discard negative frequencies (at nlen/2 and above), keep DC (at 0)}
         \PY{n}{transf} \PY{o}{=} \PY{n}{transf}\PY{p}{[}\PY{l+m+mi}{0}\PY{p}{:}\PY{n+nb}{int}\PY{p}{(}\PY{n}{nlen}\PY{o}{/}\PY{l+m+mi}{2}\PY{p}{)}\PY{p}{]}
\end{Verbatim}


    \begin{Verbatim}[commandchars=\\\{\}]
{\color{incolor}In [{\color{incolor}18}]:} \PY{c+c1}{\PYZsh{} Plot}
         \PY{n}{fig}\PY{p}{,} \PY{p}{(}\PY{n}{ax0}\PY{p}{,} \PY{n}{ax1}\PY{p}{)} \PY{o}{=} \PY{n}{plt}\PY{o}{.}\PY{n}{subplots}\PY{p}{(}\PY{n}{nrows}\PY{o}{=}\PY{l+m+mi}{2}\PY{p}{,} \PY{n}{sharex}\PY{o}{=}\PY{k+kc}{True}\PY{p}{)}
         \PY{n}{ax0}\PY{o}{.}\PY{n}{plot}\PY{p}{(}\PY{n}{nf}\PY{p}{,} \PY{n}{np}\PY{o}{.}\PY{n}{absolute}\PY{p}{(}\PY{n}{transf}\PY{p}{)}\PY{p}{)}
         \PY{n}{ax0}\PY{o}{.}\PY{n}{set\PYZus{}ylabel}\PY{p}{(}\PY{l+s+s1}{\PYZsq{}}\PY{l+s+s1}{magnitude [?]}\PY{l+s+s1}{\PYZsq{}}\PY{p}{)}
         \PY{n}{ax1}\PY{o}{.}\PY{n}{plot}\PY{p}{(}\PY{n}{nf}\PY{p}{,} \PY{n}{np}\PY{o}{.}\PY{n}{angle}\PY{p}{(}\PY{n}{transf}\PY{p}{)}\PY{p}{)}
         \PY{n}{ax1}\PY{o}{.}\PY{n}{set\PYZus{}ylabel}\PY{p}{(}\PY{l+s+s1}{\PYZsq{}}\PY{l+s+s1}{phase [rad]}\PY{l+s+s1}{\PYZsq{}}\PY{p}{)}
         \PY{n}{plt}\PY{o}{.}\PY{n}{xlabel}\PY{p}{(}\PY{l+s+s1}{\PYZsq{}}\PY{l+s+s1}{frequency [THz]}\PY{l+s+s1}{\PYZsq{}}\PY{p}{)}
         \PY{n}{plt}\PY{o}{.}\PY{n}{xlim}\PY{p}{(}\PY{l+m+mi}{0}\PY{p}{,}\PY{l+m+mi}{1000}\PY{p}{)} \PY{c+c1}{\PYZsh{} Truncate at 1THz}
         \PY{n}{plt}\PY{o}{.}\PY{n}{show}\PY{p}{(}\PY{p}{)}
\end{Verbatim}


    \begin{center}
    \adjustimage{max size={0.9\linewidth}{0.9\paperheight}}{output_33_0.png}
    \end{center}
    { \hspace*{\fill} \\}
    
    Now all that remains is to determine the reflected E-field
\(E_r(\omega)\). This field is slightly harder to calculate than \(E_i\)
and \(E_t\) as the current pulse that generates the EM pulse in the
simulation actually produces two EM pulses, one traveling towards each
end of the simulation. The forward propegating pulse is our incident
pulse (we used this to calculate \(E_i(\omega)\) previously), while the
backward propegating pulse is unwanted. Since the reference field never
interacts with a material, \(E^{i=1}_\text{ref}(t)\) will only contain
the backward propegating pulse. As such, we can determine the reflected
field \(E_r(t)\) from \[E_r(t)=E^{i=1}(t)-E^{i=1}_\text{ref}(t)\] As in
the case of determining the transmitted field \(E_t\) we can never
capture the entire transmitted E-field as there will be an infinite
number of reflections inside the material. However, as before, if we run
the simulation for a long enough time the E-field remaining in the
material will become negligable.

    \begin{Verbatim}[commandchars=\\\{\}]
{\color{incolor}In [{\color{incolor}19}]:} \PY{c+c1}{\PYZsh{} Plot the reflected E\PYZhy{}field passing through i=1}
         \PY{n}{r} \PY{o}{=} \PY{n}{els}\PY{p}{[}\PY{p}{:}\PY{p}{,}\PY{l+m+mi}{0}\PY{p}{]} \PY{c+c1}{\PYZsh{} Extract the transmitted E\PYZhy{}field}
         \PY{n}{rref} \PY{o}{=} \PY{n}{erls}\PY{p}{[}\PY{p}{:}\PY{p}{,}\PY{l+m+mi}{0}\PY{p}{]}
         \PY{n}{refl} \PY{o}{=} \PY{n}{r} \PY{o}{\PYZhy{}} \PY{n}{rref}
         \PY{n}{plt}\PY{o}{.}\PY{n}{plot}\PY{p}{(}\PY{l+m+mf}{1e3}\PY{o}{*}\PY{n}{n}\PY{p}{,} \PY{n}{np}\PY{o}{.}\PY{n}{real}\PY{p}{(}\PY{n}{refl}\PY{p}{)}\PY{p}{)} \PY{c+c1}{\PYZsh{} 1e3 scale factor converts from ps \PYZhy{}\PYZgt{} fs}
         \PY{n}{plt}\PY{o}{.}\PY{n}{xlabel}\PY{p}{(}\PY{l+s+s1}{\PYZsq{}}\PY{l+s+s1}{time [fs]}\PY{l+s+s1}{\PYZsq{}}\PY{p}{)}
         \PY{n}{plt}\PY{o}{.}\PY{n}{show}\PY{p}{(}\PY{p}{)}
\end{Verbatim}


    \begin{center}
    \adjustimage{max size={0.9\linewidth}{0.9\paperheight}}{output_35_0.png}
    \end{center}
    { \hspace*{\fill} \\}
    
    We extract \(E_t(\omega)\) via a Fourier transform as usual
\[\mathcal{F}\left\{E_r(t)\right\}=E_r(\omega)\] where again
\(\mathcal{F}\left\{A(t)\right\}=A(\omega)\) denotes a Fourier Transform
on \(A\)

    \begin{Verbatim}[commandchars=\\\{\}]
{\color{incolor}In [{\color{incolor}20}]:} \PY{c+c1}{\PYZsh{} Perform FFT}
         \PY{n}{reflf} \PY{o}{=} \PY{n}{fft}\PY{p}{(}\PY{n}{refl}\PY{p}{)}
         \PY{c+c1}{\PYZsh{} Discard negative frequencies (at nlen/2 and above), keep DC (at 0)}
         \PY{n}{reflf} \PY{o}{=} \PY{n}{reflf}\PY{p}{[}\PY{l+m+mi}{0}\PY{p}{:}\PY{n+nb}{int}\PY{p}{(}\PY{n}{nlen}\PY{o}{/}\PY{l+m+mi}{2}\PY{p}{)}\PY{p}{]}
\end{Verbatim}


    \begin{Verbatim}[commandchars=\\\{\}]
{\color{incolor}In [{\color{incolor}21}]:} \PY{c+c1}{\PYZsh{} Plot}
         \PY{n}{fig}\PY{p}{,} \PY{p}{(}\PY{n}{ax0}\PY{p}{,} \PY{n}{ax1}\PY{p}{)} \PY{o}{=} \PY{n}{plt}\PY{o}{.}\PY{n}{subplots}\PY{p}{(}\PY{n}{nrows}\PY{o}{=}\PY{l+m+mi}{2}\PY{p}{,} \PY{n}{sharex}\PY{o}{=}\PY{k+kc}{True}\PY{p}{)}
         \PY{n}{ax0}\PY{o}{.}\PY{n}{plot}\PY{p}{(}\PY{n}{nf}\PY{p}{,} \PY{n}{np}\PY{o}{.}\PY{n}{absolute}\PY{p}{(}\PY{n}{reflf}\PY{p}{)}\PY{p}{)}
         \PY{n}{ax0}\PY{o}{.}\PY{n}{set\PYZus{}ylabel}\PY{p}{(}\PY{l+s+s1}{\PYZsq{}}\PY{l+s+s1}{magnitude [?]}\PY{l+s+s1}{\PYZsq{}}\PY{p}{)}
         \PY{n}{ax1}\PY{o}{.}\PY{n}{plot}\PY{p}{(}\PY{n}{nf}\PY{p}{,} \PY{n}{np}\PY{o}{.}\PY{n}{angle}\PY{p}{(}\PY{n}{reflf}\PY{p}{)}\PY{p}{)}
         \PY{n}{ax1}\PY{o}{.}\PY{n}{set\PYZus{}ylabel}\PY{p}{(}\PY{l+s+s1}{\PYZsq{}}\PY{l+s+s1}{phase [rad]}\PY{l+s+s1}{\PYZsq{}}\PY{p}{)}
         \PY{n}{plt}\PY{o}{.}\PY{n}{xlabel}\PY{p}{(}\PY{l+s+s1}{\PYZsq{}}\PY{l+s+s1}{frequency [THz]}\PY{l+s+s1}{\PYZsq{}}\PY{p}{)}
         \PY{n}{plt}\PY{o}{.}\PY{n}{xlim}\PY{p}{(}\PY{l+m+mi}{0}\PY{p}{,}\PY{l+m+mi}{1000}\PY{p}{)} \PY{c+c1}{\PYZsh{} Truncate at 1THz}
         \PY{n}{plt}\PY{o}{.}\PY{n}{show}\PY{p}{(}\PY{p}{)}
\end{Verbatim}


    \begin{center}
    \adjustimage{max size={0.9\linewidth}{0.9\paperheight}}{output_38_0.png}
    \end{center}
    { \hspace*{\fill} \\}
    
    We can now finally calculate
\[R(\omega)=\frac{E_r(\omega)}{E_i(\omega)} \qquad T(\omega)=\frac{E_t(\omega)}{E_i(\omega)}\]

    \begin{Verbatim}[commandchars=\\\{\}]
{\color{incolor}In [{\color{incolor}22}]:} \PY{c+c1}{\PYZsh{} Calculate coefficients}
         \PY{n}{R} \PY{o}{=} \PY{n}{np}\PY{o}{.}\PY{n}{divide}\PY{p}{(}\PY{n}{reflf}\PY{p}{,} \PY{n}{incf}\PY{p}{)}
         \PY{n}{T} \PY{o}{=} \PY{n}{np}\PY{o}{.}\PY{n}{divide}\PY{p}{(}\PY{n}{transf}\PY{p}{,} \PY{n}{incf}\PY{p}{)}
         \PY{c+c1}{\PYZsh{} Plot}
         \PY{n}{fig}\PY{p}{,} \PY{p}{(}\PY{n}{ax0}\PY{p}{,} \PY{n}{ax1}\PY{p}{)} \PY{o}{=} \PY{n}{plt}\PY{o}{.}\PY{n}{subplots}\PY{p}{(}\PY{n}{nrows}\PY{o}{=}\PY{l+m+mi}{2}\PY{p}{,} \PY{n}{sharex}\PY{o}{=}\PY{k+kc}{True}\PY{p}{)}
         \PY{n}{ax0}\PY{o}{.}\PY{n}{plot}\PY{p}{(}\PY{n}{nf}\PY{p}{,} \PY{n}{np}\PY{o}{.}\PY{n}{absolute}\PY{p}{(}\PY{n}{R}\PY{p}{)}\PY{p}{,} \PY{n}{label}\PY{o}{=}\PY{l+s+s1}{\PYZsq{}}\PY{l+s+s1}{R}\PY{l+s+s1}{\PYZsq{}}\PY{p}{)}
         \PY{n}{ax0}\PY{o}{.}\PY{n}{plot}\PY{p}{(}\PY{n}{nf}\PY{p}{,} \PY{n}{np}\PY{o}{.}\PY{n}{absolute}\PY{p}{(}\PY{n}{T}\PY{p}{)}\PY{p}{,} \PY{n}{label}\PY{o}{=}\PY{l+s+s1}{\PYZsq{}}\PY{l+s+s1}{T}\PY{l+s+s1}{\PYZsq{}}\PY{p}{)}
         \PY{n}{ax0}\PY{o}{.}\PY{n}{plot}\PY{p}{(}\PY{n}{nf}\PY{p}{,} \PY{n}{np}\PY{o}{.}\PY{n}{absolute}\PY{p}{(}\PY{n}{R}\PY{p}{)}\PY{o}{+}\PY{n}{np}\PY{o}{.}\PY{n}{absolute}\PY{p}{(}\PY{n}{T}\PY{p}{)}\PY{p}{,} \PY{n}{label}\PY{o}{=}\PY{l+s+s1}{\PYZsq{}}\PY{l+s+s1}{R+T}\PY{l+s+s1}{\PYZsq{}}\PY{p}{)}
         \PY{n}{ax0}\PY{o}{.}\PY{n}{set\PYZus{}ylim}\PY{p}{(}\PY{l+m+mi}{0}\PY{p}{,}\PY{l+m+mf}{1.5}\PY{p}{)}
         \PY{n}{ax0}\PY{o}{.}\PY{n}{legend}\PY{p}{(}\PY{n}{loc}\PY{o}{=}\PY{l+m+mi}{1}\PY{p}{)}
         \PY{n}{ax0}\PY{o}{.}\PY{n}{set\PYZus{}ylabel}\PY{p}{(}\PY{l+s+s1}{\PYZsq{}}\PY{l+s+s1}{coefficient}\PY{l+s+s1}{\PYZsq{}}\PY{p}{)}
         \PY{n}{ax1}\PY{o}{.}\PY{n}{plot}\PY{p}{(}\PY{n}{nf}\PY{p}{,} \PY{n}{np}\PY{o}{.}\PY{n}{angle}\PY{p}{(}\PY{n}{R}\PY{p}{)}\PY{p}{,} \PY{n}{label}\PY{o}{=}\PY{l+s+s1}{\PYZsq{}}\PY{l+s+s1}{R}\PY{l+s+s1}{\PYZsq{}}\PY{p}{)}
         \PY{n}{ax1}\PY{o}{.}\PY{n}{plot}\PY{p}{(}\PY{n}{nf}\PY{p}{,} \PY{n}{np}\PY{o}{.}\PY{n}{angle}\PY{p}{(}\PY{n}{T}\PY{p}{)}\PY{p}{,} \PY{n}{label}\PY{o}{=}\PY{l+s+s1}{\PYZsq{}}\PY{l+s+s1}{T}\PY{l+s+s1}{\PYZsq{}}\PY{p}{)}
         \PY{n}{ax1}\PY{o}{.}\PY{n}{plot}\PY{p}{(}\PY{n}{nf}\PY{p}{,} \PY{n}{np}\PY{o}{.}\PY{n}{angle}\PY{p}{(}\PY{n}{R}\PY{p}{)}\PY{o}{+}\PY{n}{np}\PY{o}{.}\PY{n}{angle}\PY{p}{(}\PY{n}{T}\PY{p}{)}\PY{p}{,} \PY{n}{label}\PY{o}{=}\PY{l+s+s1}{\PYZsq{}}\PY{l+s+s1}{R+T}\PY{l+s+s1}{\PYZsq{}}\PY{p}{)}
         \PY{n}{ax1}\PY{o}{.}\PY{n}{legend}\PY{p}{(}\PY{n}{loc}\PY{o}{=}\PY{l+m+mi}{1}\PY{p}{)}
         \PY{n}{ax1}\PY{o}{.}\PY{n}{set\PYZus{}ylabel}\PY{p}{(}\PY{l+s+s1}{\PYZsq{}}\PY{l+s+s1}{phase [rad]}\PY{l+s+s1}{\PYZsq{}}\PY{p}{)}
         \PY{n}{ax1}\PY{o}{.}\PY{n}{set\PYZus{}xlabel}\PY{p}{(}\PY{l+s+s1}{\PYZsq{}}\PY{l+s+s1}{frequency [THz]}\PY{l+s+s1}{\PYZsq{}}\PY{p}{)}
         \PY{n}{ax1}\PY{o}{.}\PY{n}{set\PYZus{}xlim}\PY{p}{(}\PY{l+m+mi}{0}\PY{p}{,} \PY{l+m+mi}{1000}\PY{p}{)} \PY{c+c1}{\PYZsh{} Truncate at 1THz}
         \PY{n}{plt}\PY{o}{.}\PY{n}{show}\PY{p}{(}\PY{p}{)}
\end{Verbatim}


    \begin{center}
    \adjustimage{max size={0.9\linewidth}{0.9\paperheight}}{output_40_0.png}
    \end{center}
    { \hspace*{\fill} \\}
    
    There is a clear Fabry-Perot effect present in the material
transmission. Notably \(R(\omega)+T(\omega)\neq1\), which suggests
values are being incorrectly normalized somewhere in the simulation.
Hopefully further investigation will determine the cause of this issue.


    % Add a bibliography block to the postdoc
    
    
    
    \end{document}
